\subsection {Двумерное нормальное распределение}
	Двумерная случайная величина $(X, Y)$ называется распределённой нормально (или просто нормальной), если её плотность вероятности определена формулой\\ \begin{eqnarray}N(x, y, \overline x, \overline y, \sigma_x, \sigma_y, \rho) = \frac{1}{2\pi\sigma_x\sigma_y\sqrt{1 - \rho^2}} \nonumber
\\\times exp\left\{-\frac{1}{2(1 - \rho^2)}\left[ \frac{(x - \overline x)^2}{\sigma_x^2} - 2\rho\frac{(x - \overline x)(y - \overline y)}{\sigma_x\sigma_y} + \frac{(y - \overline y)^2}{\sigma_y^2}\right] \right\}\end{eqnarray}\\

	Компоненты $X, Y$ двумерной нормальной случайной величины также распределены нормально с математическими ожиданиями $\overline x, \overline y$ и средними квадратическими отклонениями $\sigma_x, \sigma_y$ соответственно [1, с. 133-134].\\

Параметр $\rho$ называется коэффициентом корреляции.

\subsection {Корреляционный момент (ковариация) и коэффициент корреляции}
	Корреляционный момент, иначе ковариация, двух случайных величин $X$ и $Y$:\\
	\begin{equation}K = cov(X, Y) = M[(X - \overline x)(Y - \overline y)].\end{equation}\\
	Коэффициент корреляции $\rho$ двух случайных величин $X$ и $Y$:\\
	\begin{equation}\rho = \frac{K}{\sigma_x\sigma_y}\end{equation}

\subsection {Выборочные коэффициенты корреляции}
	\subsubsection {Выборочный коэффициент корреляции Пирсона}
		Выборочный коэффициент корреляции Пирсона:
		\begin{equation}r = \frac{\frac{1}{n}\sum(x_i - \overline x)(y_i - \overline y)}{\sqrt{\frac{1}{n}\sum(x_i - \overline x)^2\frac{1}{n} \sum(y_i - \overline y)^2}} = \frac {K}{s_Xs_Y},\end{equation}\\ где $K, s_X^2, s_Y^2$ - выборочные ковариация и дисперсии с.в. $X$ и $Y$ [1, с. 535]

	\subsubsection {Выборочный квадратный коэффициент корреляции}
		Выборочный квадрантный коэффициент корреляции\\
		\begin{equation}r_Q = \frac{(n_1 + n_3) - (n_2 + n_4)}{n},\end{equation}\\
		где $n_1, n_2, n_3 и n_4$ - количества точек с координатами $(x_i, y_i)$, попавшими соответственно в $I, II, III и IV$ квадранты декартовой системы с осями $x^{'} = x - med x, y^{'} = y - med y$ и с центром в точке с координатами $(med x, med y)$ [1, с. 539].

	\subsubsection {Выборочный коэффицент  ранговой корреляции Спирмена}
		Обозначим ранги, соответствующие значениям переменной $X$, через $u$, а ранги, соответствующие значениям переменной $Y$, - через $v$.\\
		Выборочный коэффициент ранговой корреляции Спирмена:\\
		\begin{equation}r_S = \frac{\frac{1}{n}\sum(u_i - \overline u)(v_i - \overline v)}{\sqrt{\frac{1}{n}(u_i - \overline u)^2\frac{1}{n}\sum(v_i - \overline v)^2}},\end{equation}\\
		где $\overline u = \overline v = \frac{1 + 2 + ... + n}{n} = \frac{n + 1}{2}$ - среднее значение рангов [1, с. 540-541].

\subsection {Эллипсы рассеивания}
	Уравнение проекции эллипса рассеивания на плоскость $xOy$:\\
	\begin{equation}\frac{(x - \overline x)^2}{\sigma_x^2} - 2\rho\frac{(x - \overline x)(y - \overline y)}{\sigma_x\sigma_y} + \frac{(y - \overline y)^2}{\sigma_y^2} = const\end{equation}\\
	Центр эллипса (7) находится в точке с координатами $(\overline x, \overline y)$; оси симметрии эллипса составляют с осью $Ox$ углы, определяемые уравнением\\
	\begin{equation}
		tg \; 2\alpha = \frac{2\rho \ \sigma_x \sigma_y}{\sigma_x^2 - \sigma_y^2}
	\end{equation}

\subsection {Простая линейная регрессия}
	\subsubsection {Модель простой линейной регрессии}
		Регрессионную модель описания данных называют простой линейной регрессией, если\\
		\begin{equation} y_i = \beta_0 + \beta_1 x_i + \epsilon_i, i = 1, ..., n, \end{equation}\\
		где $x_1, ..., x_n$ - заданные числа (значения фактора); $y_1, ..., y_n$ - наблюдаемые значения отклика; $\epsilon_1, ..., \epsilon_n$ - независимые, нормально распределенные $N(0, \sigma)$ с нулевым математическим ожиданием и одинаковой (неизвестной) дисперсией случайные величины (ненаблюдаемые); $\beta_0, \beta_1$ - неизвестные параметры, подлежащие оцениванию.

	\subsubsection {Метод наименьших квадратов}
		Метод наименьших квадратов (МНК):\\
		\begin{equation}
			Q(\beta_0, \beta_1) = \sum_{i=1}^n\epsilon_i^2 = \sum_{i=1}^n(y_i - \beta_0 - \beta_1x_i)^2 \rightarrow \min_{\beta_0, \beta_1}.
		\end{equation}

	\subsubsection {Расчетные формулы для МНК-оценок}
		МНК-оценки параметров $\beta_0$ и $\beta_1$:\\
		\begin{equation} \hat\beta_1 = \frac{\overline{xy} - \overline x \cdot \overline y}{\overline{x^2} - (\overline x)^2}, \end{equation}
		\begin{equation} \hat\beta_0 = \overline y - \overline x \hat\beta_1. \end{equation}

\subsection {Робастные оценки коэффицентов линейной регрессии}
	Метод наименьших модулей:\\
	\begin{eqnarray} 
		\sum_{i=1}^n|y_i - \beta_0 - \beta_1x_i| \rightarrow \min_{\beta_0, \beta_1}. \\
		\hat\beta_{1R} = r_Q\frac{q_y^*}{q_x^*}, \\
		\hat\beta_{0R} = med \; y - \hat\beta_{1R} med \; x, \\
		r_Q = \frac{1}{n}\sum_{i=1}{n}sgn(x_i - med \; x) sgn(y_i - med \; y), \\
		q_y^* = \frac{y_{(j)} - y_{(l)}}{k_q(n)}, \; q_x^* = \frac{x_{(j) - x_{(l)}}}{k_q(n)}.
	\end{eqnarray}\\
	$$l = \left\{\begin{array}{cccc}
			[n/4] + 1 & \text{при} & n/4 & \text{дробном}, \\
			 n/4 & \text{при} & n/4  & \text{целом}.
			\end{array}\right.$$\\
	$$l = n - l + 1$$\\
	$$sgn \; z = \left\{\begin{array}{ccc}
					1 & \text{при} & z > 0, \\
					0 & \text{при} & z = 0, \\ 
					-1 & \text{при} & z < 0.
					\end{array}\right.$$\\
	Уравнение регрессии здесь имеет вид\\
	\begin{equation}y = \hat\beta_{0R} + \hat\beta_{1R}x.\end{equation}
				 

\subsection {Метод максимального правдоподобия}
	$L(x_1, ..., x_n, \theta)$ - функция правдоподобия (ФП), рассматриваемая как функция неизвестного параметра $\theta$:\\
	\begin{equation} L(x_1, ..., x_n, \theta) = f(x_1, \theta) f(x_2, \theta) ... f(x_n, \theta) \end{equation}\\
	Оценка максимального правдоподобия:\\
	\begin{equation} \hat\theta_{мп} = ard \; \max_\theta L(x_1, ..., x_n, \theta). \end{equation}\\
	Система уравнений правдоподобия (в случае дифференцируемости функции правдоподобия):\\
	\begin{equation} \frac{\partial L}{\partial \theta_k} = 0 \; \text{или} \; \frac{\partial ln L}{\partial \theta_k} = 0, \; k = 1, ..., m.\end{equation}

\subsection {Проверка гипотезы о законе распределения генеральной совокупности. Метод хм-квадрат}
	Выдвинута гипотеза $H_0$ о генеральном законе распределения с функцией распределения $F(x)$.\\
	Рассматриваем случай, когда гипотетическая функция распределения $F(x)$ не содержит неизвестных параметров.\\
	Правило проверки гипотезы о законе распределения по методу $\chi^2$:\\
	\begin{enumerate}
		\item {Выбираем уровень значимости $\alpha$}
		\item {По таблице [3, с. 358] находим квантиль $\chi^2_{1 - \alpha}(k - 1)$ распределения хи-квадрат с  $k - 1$ степенями свободы порядка $1 - \alpha$.}
		\item {С помощью гипотетической функции распределения $F(x)$ вычисляем вероятности $p_i = P(X\in \Delta_i), i = 1, ..., k$.}
		\item {Находим частоты $n_i$ попадания элементов выборки в подмножества $\Delta_i, i = 1, ..., k.$}
		\item {Вычисляем выборочное значение статистики критерия $\chi^2$:\\
			$$\chi^2_B = \sum_{i=1}^k\frac{(n_i - np_i)^2}{np_i}.$$}
		\item {
		Сравниваем $\chi^2_B$ и квантиль $\chi^2_{1 - \alpha}(k - 1)$.\\
		\begin{enumerate}
			\item {Если $\chi^2_B < \chi^2_{1 - \alpha}(k - 1)$, то гипотеза $H_0$ на данном этапе проверки принимается.}
			\item {Если $\chi^2_B \geq \chi^2_{1 - \alpha}(k - 1)$, то гипотеза $H_0$ отвергается, выбирается одно из альтернативных распределений, и процедура проверки повторяется.}
		\end{enumerate}}
	\end{enumerate}

\subsection {Доверительные интервалы для параметров нормального распределения}
	\subsubsection {Доверительный интервал для математического ожидания $m$ нормального распределения}
		Дана выборка $(x_1, x_2, ..., x_n)$ объёма $n$ из нормальной генеральной совокупности. На её основе строим выборочное среднее $\overline x$ и выборочное среднее квадратическое отклонение $s$. Параметры $m$ и $\sigma$ нормального распределения неизвестны.\\
		Доверительный интервал для $m$ с доверительной вероятностью $\gamma = 1 - \alpha$:\\
		\begin{eqnarray}
			P\left(\overline x - \frac{sx}{\sqrt{n - 1}} < m < \overline x + \frac{sx}{\sqrt{n - 1}}\right) = 2F_T(x) - 1 = 1 - \alpha, \\ \nonumber
			P\left(\overline x - \frac{st_{1 - \alpha/2}(n - 1)}{\sqrt{n - 1}} < m < \overline x + \frac{st_{1 - \alpha/2}(n - 1)}{\sqrt{n - 1}} \right) = 1 - \alpha,
		\end{eqnarray}

	\subsubsection {Доверительный интервал для среднего квадратического отклонения $\sigma$ нормального распределения}
		Дана выборка $(x_1, x_2, ..., x_n)$ объёма $n$ из нормальной генеральной совокупности. На её основе строим выборочную дисперсию $s^2$. Параметры $m$ и $\sigma$ нормального распределения неизвестны.\\
		Задаёмся уровнем значимости $\alpha.$\\
		Доверительный интервал для $\sigma$ с доверительной вероятностью $\gamma = 1 - \alpha$:\\
		\begin{equation} P\left( \frac{s\sqrt{n}}{\sqrt{\chi^2_{1-\alpha/2}(n - 1)}} < \sigma < \frac{s\sqrt{n}}{\sqrt{\chi^2_{\alpha/2}(n - 1)}}\right) = 1 - \alpha, \end{equation} 

\subsection {Доверительные интервалы для математического ожидания $m$ и среднего квадратического отклонения $\sigma$ произвольного распределения при большом объеье выборки. Асимптотический подход}
	При большом объёме выборки для построения доверительных интервалов может быть использован асимптотический метод на основе центральной предельной теоремы.

	\subsubsection {Доверительный интервал для математического ожидания $m$ произвольной генеральной совокупности при большом объеме выборки}
	Предполагаем, что исследуемое генеральное распределение имеет конечные математическое ожидание $m$ и дисперсию $\sigma^2.$\\
	$u_{1-\alpha/2}$ - квантиль нормального распределения $N(0, 1)$ порядка $1 - \alpha/2$.\\
	Доверительный интервал для $m$ с доверительной вероятностью $\gamma = 1 - \alpha$:\\
	\begin{equation} P\left(\overline x - \frac{su_{1 - \alpha/2}}{\sqrt{n}} < m < \overline x + \frac{su_{1 - \alpha/2}}{\sqrt{n}}\right) \approx \gamma\end{equation}

	\subsubsection {Довверительный интервал для среднего квадратического отклонения $\sigma$ произвольной генеральной совокупности при большом объеме выборки}
	Предполагаем, что исследуемая генеральная совокупность имеет конечные первые четыре момента.\\
	$u_{1 - \alpha/2}$ - квантиль нормального распределения $N(0, 1)$ порядка $1 - \alpha/2$.\\
	$E = \frac{\mu_4}{\sigma^4} - 3$ - эксцесс генерального распределения, $e = \frac{m_4}{s^4}-3$ - выборочный эксцесс; $m_4 = \frac{1}{n}\sum_{i=1}^{n}(x_i - \overline x)^4$ - четвёртый выборочный центральный момент.\\
	\begin{equation} s(1 + U)^{-1/2} < \sigma < s(1 - U)^{-1/2}, \end{equation}\\
	или\\
	\begin{equation} s(1 - 0.5U) < \sigma < s(1 + 0.5U) \end{equation}\\
	где $U = u_{1-\alpha/2}\sqrt{(e + 2)/n}$\\
	Формулы (25) или (26) дают доверительный интервал для $\sigma$ с доверительной вероятностью $\gamma = 1 - \alpha$ [1, с. 461-462].\\
	Замечание. Вычисления по формуле (25) дают более надёжный результат, так как в ней меньше грубых приближений.
