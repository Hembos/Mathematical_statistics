\subsection{Выборочные коэффициенты корреляции и эллипсы рассеивания}
	Для двумерного нормального распределения дисперсии выборочных коэффициентов корреляции упорядочены следующим образом:  $r < r_S < r_Q$\\
	
	Для смеси нормальных распределений дисперсии выборочных коэффициентов корреляции упорядочены следующим образом: $r_Q < r_S < r$.\\

	Процент попавших элементов выборки в эллипс рассеивания примерно равен его теоретическому значению (95\%)

\subsection{Оценки коэффициентов линейной регрессии}
	Критерий наименьших квадратов точнее оценивает коэффициенты линейной регрессии на выборке без возмущений
	
	Критерий наименьших модулей точнее оценивает коэффициенты линейной регрессии на выборке с возмущениями. Критерий наименьших модулей устойчив к редким выбросам

\subsection{Проверка гипотезы о законе распределения генеральной совокупности. Метод хи-квадрат}
	По результатам проверки на близость с помощью критерия хи-квадрат можно принять гипотезу $H_0$ о нормальном распределении $N(x, \hat\mu, \hat\sigma)$ на уровне значимости $\alpha = 0.05$ для выборки, сгенерированной согласно $N(x, 0, 1)$.

	Видим также, что критерий принял гипотезу $H_0$ для выборок, сгенерированных по равномерному закону и закону распределения Лапласа.

	По исследованию на чувствительность видим, что при небольших объемах выборки уверенности в полученных результатах нет, критерий может ошибиться. Это обусловлено тем, что теорема Пирсона говорит про асимптотическое распределение, а при малых размерах выборки результат не будет получаться достоверным

\subsection{Доверительные интервалы для параметров распределения}
	Генеральные характеристики $(m = 0, \sigma = 1)$ накрываются построенными доверительными интервалами

	Доверительные интервалы, полученные по большей выборке, являются соответственно более точными, т.е. меньшими по длине

	Доверительные интервалы для параметров нормального распределения более надёжны, так как основаны на точном, а не асимптотическом распределении
