\subsection {Рассматриваемые распределения}
	Плотности:
	\begin {itemize}
		\item {Нормальное распределение\\ \begin{equation}N(x, 0, 1) = \frac{1}{\sqrt{2\pi}}e^{-\frac{x^2}{2}}\end{equation}}
		\item {Распределение Коши\\ \begin{equation}C(x, 0, 1) = \frac{1}{\pi}\frac{1}{x^2+1}\end{equation}}
		\item {Распределение Лапласа\\ \begin{equation}L(x, 0, \frac{1}{\sqrt{2}}) = \frac{1}{\sqrt{2}}e^{-\sqrt{2}|x|}\end{equation}}
		\item {Распределение Пуассона\\ \begin{equation}P(k, 10) = \frac{10^k}{k!}e^{-10}\end{equation}}
		\item {Равномерное распределение\\ \begin{equation}U(x, -\sqrt{3}, \sqrt{3}) = 
										\begin{cases}
											\frac{1}{2\sqrt{3}} \text{ при } |x| \leq \sqrt{3} \\
											0  \text{ при } |x| > \sqrt{3}
										\end{cases}\end{equation}}
	\end {itemize}

\subsection {Гистограмма}
	\subsubsection {Построение гистограммы}
		Множество значений, которое может принимать элемент выборки, разбивается на несколько интервалов. Чаще всего эти интервалы берут одинаковыми, но это не является строгим требованием. Эти интервалы откладываются на горизонтальной оси, затем над каждым рисуется прямоугольник. Если все интервалы были одинаковыми, то высота каждого прямоугольника пропорциональна числу элементов выборки, попадающих в соответствующий интервал. Если интервалы разные, то высота прямоугольника выбирается таким образом, чтобы его площадь была пропорциональна числу элементов выборки, которые попали в этот интервал [1].
	
	\subsubsection {Вариационный ряд}
		Вариационным ряд - последовательность элементов выборки, расположенных в неубывающем порядке. Одинаковые элементы повторяются [2, с. 409].

\subsection {Выборочные чиловые характеристики}
	\subsubsection {Характеристики положения}
		\begin{itemize}
			\item {Выборочное среднее\\ \begin{equation}\overline{x} = \frac{1}{n}\sum_{i=1}^{n}x_i\end{equation}}

			\item {Выборочная медиана\\ \begin{equation}med\ x = 
									\begin{cases}
										x_{(l+1)} \text{ при } n = 2l + 1 \\
										\frac{x_{(l)} + x_{(l+1)}}{2} \text{ при } n=2l
									\end{cases}\end{equation}}

			\item {Полусумма экстремальных выборочных элементов \\ \begin{equation}z_R =  \frac{x_{(1)}+x_{(n)}}{2}\end{equation}}

			\item {Полусумма квартилей\\
			Выборочная квартиль $z_p$ порядка $p$ определяется формулой \\ 
			\begin{equation}z_p = \begin{cases} x_{([np]+1)} \text{ при } np \text{ дробном,} \\ x_{(np)} \text{ при } np \text{ целом.} \end{cases}\end{equation}\\
			Полусумма квартилей\\
				\begin{equation}z_Q = \frac{z_{1/4} + z_{4/4}}{2}\end{equation}}

			\item {Усечённое среднее \\ \begin{equation}z_{tr} = \frac{1}{n-2r}\sum_{i=r+1}^{n-r}x_{(i)},\ r\approx \frac{n}{4}\end{equation}}
		\end{itemize}
	
	\subsubsection {Характеристики рассеяния}
		Выборочная дисперсия \\
		\begin{equation}D = \frac{1}{n}\sum_{i=1}^{n}(x_i-\overline{x})^2\end{equation}

\subsection {Боксплот Тьюки}
	\subsubsection {Построение}
		Границами ящика – первый и третий квартили, линия в середине ящика — медиана. Концы усов — края статистически значимой выборки (без выбросов). Длина «усов»:\\
			\begin{equation}X_1 = Q_1 - \frac{3}{2}(Q_3 - Q_1),\ X_2 = Q_3 + \frac{3}{2}(Q_3 - Q_1)\end{equation} \\ 
		где $X_1$ - нижняя граница уса, $X_2$ - верхняя граница уса, $Q_1$ - первый квартиль, $Q_3$ - третий квартиль.\\
			Данные, выходящие за границы усов (выбросы), отображаются на графике
в виде маленьких кружков [3].

	\subsubsection {Теоретическая вероятность выбросов}
		Можно вычислить теоретические первый и третий квартили распределений - $Q_1^T$ и $Q_3^T$. По ф-ле (15) – теоретические нижнюю и верхнюю границы уса $X_1^T$ и $X_2^T$.  Выбросы – величины $x$:\\
	\begin{equation}
		\left[
			\begin{gathered}
				x < X_1^T \\
				x > X_2^T
			\end{gathered}
		\right.
	 \end{equation}
	Для теоретиеских распределений:\\
	\begin{itemize}
		\item {Для непрерывных распределений\\ 
				\begin{equation} P_B^T = P(x < X_2^T) + P(x > X_2^T) = F(X_1^T) + (1 - F(X_2^T)) \end{equation}}
		
		\item {Для дискретных распределений\\
				\begin{equation} P_B^T = P(x < X_2^T) + P(x > X_2^T) = (F(X_1^T) - P(x = X_1^T)) + (1 - F(X_2^T)) \end{equation}}
	\end{itemize}


\subsection {Эмпирическая функция распределения}
	\subsubsection {Статистический ряд}
		Статистический ряд - последовательность упорядоченных по возрастанию различных элементов выборки $z_1, z_2, ..., z_k$ и и частот, с которыми эти элементы встречаются в выборке $n_1, n_2, ..., n_k$. 

	\subsubsection {Эмпирическая функция распределения}
		Эмпирическая функция распределения сопоставляет числу $x$ относительную частоту события $X<x$, полученную по данной выборке:\\
	\begin{equation} F_n^* = P^*(X<x)\end{equation}

	\subsubsection {Нахождение э. ф. р.}
		Для получения относительной частоты $P^*(X<x)$ просуммируем в статистическом ряде, построенном по данной выборке, все частоты $n_i$, для которых элементы $z_i$ статистического ряда меньше $x$. Тогда $P^*(X<x) = \frac{1}{n} \sum_{z_i<x}n_i$. Получаем\\
	\begin{equation} F^*(x) = \frac{1}{n}\sum_{z_i<x} n_i\end{equation}

\subsection {Оценки плотности вероятности}
	\subsection {Определение}
		Оценкой плотности вероятности $f(x)$ называется функция $\hat{f}(x)$, построенная на основе выборки, приближённо равная $f(x)$\\
		\begin{equation} \hat{f}(x) \approx f(x) \end{equation}

	\subsection {Ядерные оценки}
		Представим оценку в виде суммы с числом слагаемых, равным объёму выборки: \\
		\begin{equation} \hat{f}_n(x) = \frac{1}{nh_n}\sum_{i=1}^{n}K(\frac{x-x_i}{h_n})\end{equation}
		Здесь функция $K(u)$называемая ядерной (ядром), непрерывна и является плотностью вероятности, $x_1, x_2, ..., x_n$ - элементы выборки, $\{h_n\}$ - любая последовательность положительных чисел, обладающая свойствами\\
	\begin{equation} h_n \underset{n\to \infty}{\longrightarrow} 0;\ \ \ \frac{h_n}{n^{-1}} \underset{n \to \infty}{\longrightarrow} \infty \end{equation}\\
	Гауссово (нормальное) ядро [4, с. 38]\\
	\begin{equation} K(u) = \frac{1}{\sqrt{2\pi}}e^{-\frac{u^2}{2}}\end{equation}\\
	Правило Сильвермана [4, с. 44]\\
	\begin{equation} h_n = 1.06\hat{\sigma}n^{-1/5}\end{equation}\\
	где $\hat{\sigma}$ - выборочное стандартное отклонение.
